\documentclass[10pt]{article}
\usepackage{fullpage}
\textheight=9.0in
\pagestyle{empty}
%\raggedbottom
%\raggedright
\setlength{\tabcolsep}{0in}
\begin{document}

\begin{tabular*}{6.5in}{l@{\extracolsep{\fill}}r}
{\large \textbf{Dibyendu's Interests, Goals and ...}}  & \\
1424 Hopedale Ct & 678-520-5524 \\
Lawrenceville, GA 30043 &  dbaksi@gmail.com \\
\end{tabular*}
\\
\vspace{0.35 in}

{\large \textbf{Summary :}}
\vspace{0.2 in}

\subsection{Pro Stuff : }
The main ideas of interest are around distributed and decentralized control and computation. The problem of Witsenhausen Counterexample has remained unsolved for more than four decades but it requires concepts from Control Theory, Optimization and Computational Complexity. A report on various approaches and their connections to solve this outstanding research problem may lead to future insights by researchers. My goal in 2016 3rd quarter is to create a simple explanation, tutorial with running code and a presentation deck with youtube video. On the other hand, my professional career of two decades has revolved around the practice of large scale software architecture and design. The major recent developments of multicore processors and networking has led to a surge in interest in using cloud providers by all enterprises. However, this will require expertise in architecting, designing and programming distributed and parallel systems using the new paradigms of software development. At heart of this is evolution is new models of distributed computation such as CSP, CCS, $\pi$-calculus, bigraphs, actors etc using new programming languages and paradigms. 
\begin{itemize}
\item \textbf{Zone : Core Professional for Survival -} Enterprise Architecture,  Software Architecture, Infrastructure/Cloud Architecture, Design Patterns, Software Engineering Practice, Core Programming Paradigms, Databases, Information Retrieval, Machine Learning Stack and Networking.
\item \textbf{Zone : Core Domain of Economics and Finance -} Basic Economic Models, Finance Core from Mehrling, Banking, Payment Systems and Blockchain derivatives.
\item \textbf{Zone : Product Dev - } Use graph database, latest Dev tools, new language to create a CMDB and a architecture depedency management tool all the way to code level views; SE evolution demo for all layers of the stack.
\item \textbf{Zone : Distributed Computation - }  Models, Algorithms and Computation : Models of computaiton from Turing Machine, Lambda Calculus, CCS, $\pi$-calculus, Actors etc.; Computational Complexity, Algorithms, Distributed Computation, Logic, Category Theory, Formal methods applications. In the context of cryptocurrency, distributed ledger technology is one of the recent phenomenon that has generated tremendous interest in both the right and the left wing of the spectrum. Academically inclined folks are also very excited as it brings together some of the most interesting developments of fundamental Computer Science, namely, (i) distributed computation models, (ii) distributed algorithms, (iii) peer-to-peer networks, (iii) graph theory influenced combinatorial optimization problems, (iv) complexity of cryptography including things like zero knowledge proofs, (v) distributed programming languages etc. Combined with the financial transactions domain of payments of all sorts and the push to eliminate inefficiencies of financial intermediaries, there are few areas with far reaching consequences of this technology other than energy markets.
\item \textbf{Zone : Outstanding problems} : Understand Witsenhausen Counterexample state of the art, Computational Complexity connection. 
\end{itemize}
\subsection{Interests : }
\begin{itemize}
\item \textbf{Zone :  Physics and EE/Energy -} From first principles derive equations representing phenomena and energy sustainability; understand YM-MGH Claymath problem.
\item \textbf{Zone : Biology and Healthcare -} Life, evolution, gene and future of healthcare using technology writings.
\item \textbf{Zone : Automobiles and Flight -} Evolution of automobiles, Evolution and design of airplanes, launch vehicles and spacecrafts.
\end{itemize}

\vspace{0.35 in}
{\Large \textbf{Core Books :}}
\vspace{0.2 in}
\begin{enumerate}
\item Ajay Khemkalyani and Mukesh Singhal, \textit{``Distributed Computation"},  2009.
\item M. Herlihy and Nir Shavit, \textit{``Art of Multiprocessor Programming''}, 2012.
\end{enumerate}
 
 \section{Algebra }
 \begin{itemize}
 \item If $A \propto B$ when C is constant and $B \propto C$ when B is constant, then $A \propto B \cdot C$ when both B and C vary.
 \item 
 \begin{itemize}
 \item Arithmetic progression : Term n is $t_n = a + (n-1) \cdot d$; Sum of n numbers is $S_n = \frac {n}{2} \cdot \{2 \cdot a + (n-1) \cdot d\}$ where a is the first term and d, the difference between terms.
 \item Sum of $n$ numbers is $\sum_{n=1}^{\infty}n = \frac{n \cdot (n+1)}{2}$; $\sum_{n=1}^{\infty}n^2 = \frac{n \cdot (n+1) \cdot (2 \cdot n +1)}{6}$; $\sum_{n=1}^{\infty}n^3 = \{\frac{n \cdot (n+1)}{2}\}^2$
 \end{itemize}
 \end{itemize}  

\vspace{0.35 in}
{\Large \textbf{Core Papers :}}
\vspace{0.2 in}
\begin{enumerate}

\item Alan Turing,  \textit{``On Computability and..."}, Journal of the..., 1936.

\end{enumerate}

\end{document}


